% Implementierung

\chapter{Implementierung}
\label{implementierung}

\section{Aufbau Struktur des Projektes}

Wie im Kapitel \ref{konzept} erläutert stellt mARt jeweils eine zwei- und dreidimensionale Darstellung der MRT-Daten zur Verfügung, zwischen denen gewechselt werden kann. Auf Grund der beschriebenen Hindernisse bezüglich der grafischen Leistungsfähigkeit der Hololens, wird die Anwendung sowohl für die Hololens als auch als VR-Anwendung implementiert. Die AR-Anwendung bietet dabei nicht die volle Funktionalität und stellt die dreidimensionale Visualisierung nur beispielhaft dar. Die VR-Anwednung ermöglich dagegen eine interaktive Echtzeit-Darstellung des Volumens. In beiden Fällen handelt es sich allerdings im Kern um die selbe Anwendung, d.h. die Software sollte für beide Gräte entwickelt werden. 
Sowohl für die Entwicklung von VR- als auch AR-Anwendungen werden im Allgemeinen GameEngines verwendet, da sie als Oberfläche diesen, um interaktive 3D-Anwendungen zu erzeugen, die in Echtzeit funktionieren. 
zur Implementierung von mARt wurde \textit{Unity 3D} (Ref) verwendet. Die Engine gehört zu den meist genutzten und bekanntesten (Ref?) auf dem Markt und ermöglicht die Entwicklung von Software für die meisten Plattformen. Unity bietet viele nützliche Funktionen zur Entwicklung von interaktiven Anwendungen und wird außerdem von einer großen  Gemeinschaft genutzt, sodass neben einer detaillierten Dokumentation auch über Foren und Webartikel hilfreiche Informationen zur Entwicklung zur Verfügung stehen.

Microsoft empfiehlt weiterhin Unity zur Entwicklung für Hololens-Software zu Nutzen und bietet mit HoloToolkit ein Framework, das Hololens Funktionen innerhalb von Unity zur Verfügung stellt. 
%https://docs.microsoft.com/en-us/windows/mixed-reality/install-the-tools
Unity verwendet C\# als Skriptsprache, was vor allem den Einstieg in die Programmierung erleichtert. Die Engine selbst ist allerdings in C++ geschrieben, wodurch effiziente Berechnungen zur Laufzeit ermöglicht werden.
%Wieso?

Die Anwendung wird in Unity entwickelt und dann für die Hololens und als VR-Anwendung gebaut.


\subsection{Kombination von Leap Motion mit Vive/Hololens}
\subsection{Unity Projekt}

Unity Projekte basieren auf 3D-Szenen. Innerhalb einer Szene können Gameobjects platziert werden. Dabei kann es sich z.B. um 3D-Modelle handeln. Ein GameObject besitzt verschiedene Components, die dessen Eigenschaften und Verhalten bestimmen. Viele Funktionen und Eigenschaften werden von bereits in der Engine vorhandenen Components realisiert. Sogenannte Rigidbodys verleihen einem Objekt beispielsweise physikalische Eigenschaften, die von der Engine berechnet werden. Components können auch Skripte sein, die der Entwickler selbst verfasst hat. Über diese Skripte wird die Spiellogik und die Funktionalität der Anwendung definiert. 

Da mARt sich hauptsächlich in die beiden Szenarien einer zwei- und dreidimensionalen Darstellung unterteilen lässt existiert für jedes Szenario eine Szene.
Viele der Funktionen sind allerdings ähnlich oder gleich, weshalb manche Skripte in beiden Szenen verwendet werden. 

\section{Interaktion Hololens}

Im Rahmen der Arbeit wurde neben der endgültigen Anwendung zuerst auch eine Prototypische Demo-Anwendung entwickelt, die die Hololens eigenen Gesten genutzt hat. Die Anwendung hat eine zweidimensionale Darstellung der MRT-Bilder als Hologramm in den Raum projiziert, welches der Nutzer durch Hololens eigene Gesten manipulieren konnte. Dabei wurden alle vorhandenen Manipulierungsformen eines Hologramms abgedeckt, sowie das Scrollen durch die Bildschichten. Um die Hololens Gesten zu Nutzen wurde das HoloToolkit verwendet. 
Anhand dieser Demo sollte geprüft werden, ob die Interaktionsmöglichkeiten, die die Hololens bietet ausreichend sind, um einem Neurologen die effektive Untersuchung von MRT-Bildern zu ermöglichen.
Weiterhin konnten durch das Testen einer realen Anwendung die Erwartungen und Anforderungen eines Arztes an mARt noch weiter spezifiziert werden.  

Wie im Kapitel \ref{konzept}erläutert haben sich die Gesten der Hololens zwar als ausreichend erwiesen, machten aber die Verwendung von weiteren Bedienelementen notwendig, z.B. zum Wechseln der Manipulationsart. Aus den beschriebenen Gründen wurde für die endgültige Implementierung der Anwendung die Motion Leap in das System integriert.
% Genauer auf Implementierung eingehen

\subsection{HoloToolkit}

\section{Interaktion Vive}
\label{3dscene}

- Aufbau Szene
- Funktionsweise der GameObjekte/ Skripte

\subsection{Leap Motion}

Damit die Motion Leap Kamera in einer Anwendung verwendet werden kann, muss das von \textit{Motion Leap} zur Verfügung gestellte SDK eingebunden werden. Das \textit{Core SDK} ist dabei zur Erkennung und Verwendung des Gerätes notwendig, während das das erweiternde Paket für Interaktionen die Funktionen und Beispiele bietet, wie verschiedene Bedienelemente in eine Anwendung integriert werden können, die auf die Controller, also die Hände des Nutzers reagieren.
Meistens funktioniert dies indem einzelne Skripte, die das gewünschte Verhalten implementieren als Komponenten an Spieleobjekte angehangen werden. Die Skripte lösen dann bestimmte Events oder Methoden aus, die mit den Funktionalitäten der Anwendung verknüpft werden. 

\section{Shader in Unity}

Das Volume Rendering der MRT-Daten wird innerhalb eines Shaders berechnet. An dieser Stelle soll ein Überblick über die Funktionsweise und den Aufbau von Unity-Shadern gegeben werden. 

Wie ein Objekt in einer Szene gerendert wird hängt in Unity davon ab, welches Material diesem zugewiesen ist. Das Material fungiert dabei als eine Art Behälter für sämtliche Parameter, die das Aussehen des Objektes beeinflussen, wie z.B. die Textur oder Farbe. Welche Parameter das Material besitzt und wie diese miteinander verrechnet werden bestimmt der Shader des Materials. Innerhalb des Shaders wird in Abhängigkeit zu den diesem übergebenen Werten die Farbe für jeden Pixel errechnet. 
% https://docs.unity3d.com/Manual/Shaders.html

Shader in Unity sind in der Unity eigenen Shader-Sprache \textit{Shader Lab} geschrieben. Im Shader ist definiert, welche Eigenschaften dieser besitzt und welche Sub- und Fallback-Shader er verwendet.
Die Eigenschaften sind die eben genannten Parameter, dessen Werte dann über das Material gesetzt werden. Hier werden deren Namen, Typen, ihr Wertebereich, sowie ihre Standartwerte definiert. 

Schließlich werden im Shader auch sogenannte Subshader definiert, die den eigentlichen Shader-Code enthalten.
Ein Shader kann mehrere Subshader enthalten für den Fall, dass einer der Shader von einem Gerät nicht unterstützt wird. Wird keiner unterstützt wird der Fallback-Shader verwendet. 
Neben dem Shader-Code besitzt ein Subshader Tags, die bestimmen, wann und wie ein Shader von der Rendering Engine gerendert werden soll. Dies kann sich beispielsweise auf die Reihenfolge beziehen, in der Objekte gerendert werden oder ob ein Objekt Schatten werfen soll. 
Nach den Tags folgt die Definition eines \textit{Pass}. Als Render Pass wird der gesamte Prozess bezeichnet, der durchlaufen wird, um einen Pixel zu rendern. Angefangen bei der Berechnung einzelner Vertices eines Meshes über den Vertex-Shader bis zum Fragment-Shader. D.h. im Pass werden in verschiedenen Methoden die tatsächlichen Berechnungen beschrieben, die zum Aussehen jedes Pixels führen. 
Auch ein Pass kann zunächst wieder Tags enthalten, die bestimmen wann oder wie oft ein Pass durchlaufen werden soll. 
% Render setup ????
Dann folgt der Code Abschnitt, der den Shader-Code enthält. Dieser ist in Cg (C for Graphics) geschrieben, einer Shading-Sprache syntaktisch stark HLSL (High-Level Shading Language). 
Abhängig davon, um welche Art von Shader es sich handelt werden hier die notwendigen Funktionen implementiert. Unity besitzt sogenannte Surface-Shader, ein vereinfachter Shader, für den kein Beleuchtungsmodell implementiert werden muss (?). Andererseits können auch Vertex-Fragment-Shader implementiert werden, die auch Unlit-Shader heißen. 

Das Volume Rendering der MRT-Daten erfolgt durch einen Vertex-Fragment-Shader auf die genaue Implementierung wird im Folgenden eingegangen. 

% https://docs.unity3d.com/Manual/SL-Shader.html
-Was is cg hlsl?
- Was sind compute, surface, unlit shader?
- Graphic pipeline?


\section{Volume Rendering}
- Grundlage: GitHub Projekt

Wie in Kapitel \ref{konzept} beschrieben, wurde der 3D-Darstellung des Gehirns mit Hilfe von Volume Rendering umgesetzt. 

Im Kapitel \ref{grundlagen} wurde der theoretische Vorgang dieser Technik beschrieben. Dieses Kapitel fokussiert sich auf die Implementierung der einzelnen Schritte.

Die volumetrische Darstellung des Gehirns wird im Grunde genommen mit nur 3 Skripten erzeugt. Zuerst wird in VolumeRendering.cs das Mesh des Würfeln generiert. Hier werden auch die Parameter aktualisiert, die für das Rendering relevant sind, wie z.B. die 3D-Textur oder Farbe, sowie die Parameter, die durch die Nutzerschnittstelle manipuliert werden können. 
Die Parameter werden an den Shader übergeben, in dem das Rendering definiert ist. Der Cg/HLSL Code, der den Vertex- und Fragmentshader implementiert ist dabei in ein eigenes Script ausgelagert.
% WARUM??

\subsection{Ray Casting}

Im Fragmentshader wird zunächst ein Strahl definiert, der von dem aktuell betrachteten Vertex aus von der Kameraposition in die Welt "geschossen" wird. In einem selbst definierten struct werden die maximalen und minimalen Werte definiert, aus denen sich die Eckpunkte des dargestellten Würfeln zusammensetzten. Anschließend wird geprüft, ob der Strahl den Würfel schneidet. 

%-----------Intersect-----------
%Dies geschieht indem zuerst die Inverse der Strahlrichtung ermittelt wird. Die Inverse eines Vektors $v$ ist $v^-1$ und da $v^-x=\frac1}{v^x}$, ergibt sich die Inverse des Richtungsvektors des Strahls indem $1$ durch die dividiert wird.
% AABB, axis aligend etc? intersection formula REFERENZ
Um die Schnittpunkte des Strahls zu ermitteln nimmt man an, dass die sechs Seiten des Würfels auf jeweils sechs Ebenen liegen, wobei davon zwei immer parallel sind. Zuerst werden jetzt alle Schnittpunkte des Strahls mit diesen Ebenen berechnet und dann geprüft, ob die Schnittpunkte innerhalb des Würfels liegen.
Der Würfel wird durch zwei Eckpunkte beschrieben. Da der Würfel Koordinaten von -0,5 bis 0,5 hat können wir hierfür die jeweils kleinsten und größten Koordinaten nutzen. 
Die Schnittpunkte des Strahls, mit den x-, y- und z-Ebenen ergeben sich durch das Umstellen der Formel, die einen Strahl beschreibt:

$p=r_{Ursprung}+t*r_{Richtung}$

$r_{Ursprung}$ ist dabei der Ursprung des Strahls und $r_{Richtung}$ seine Richtung. $p$ ist ein Punkt auf dem Strahl und $t$ ein Parameter, der bestimmt wie weit der Punkt vom Ursprung entfernt ist.

Um die Schnittpunkte mit den Ebenen zu erhalten wird die Formel nach $t$ umgestellt:

$t=(p-r_{Ursprung})/r_{Richtung}$

Für $p$ werden jeweils die beiden Eckpunkte des Würfels eingesetzt, die den Würfel beschreiben. Dadurch sind in den zwei dreidimensionalen Vektoren $t_{unten}$ und $t_{oben}$ insgesamt 6 Schnittpunkte mit den Ebenen bekannt. Zwei auf jeder Ebene. Durch den Vergleich der t-Vektoren wird festgelegt, welcher Eckpunkt (und damit welche Ebene) weiter vorne liegt. 
% Was wenn Strahl parallel zur Ebene ist??
Jetzt muss bestimmt werden, ob diese Schnittpunkte sich innerhalb des Würfels befinden.
Dazu werden jeweils die x-, y- und z-Werte der t-Vektoren untereinander verglichen. Für den näher gelegenen t-Vektor wird der maximale bestimmt, für den weiter entfernten der minimale Wert bestimmt. Ist der Wert des näheren ts größer als der des entfernten, liegt der Schnittpunkt nicht in dem Würfel. Andersherum tut er es.

%----------------------------------------
Die beiden t-Werte werden als $t_{nah}$ und $t_{fern}$ gespeichert.
Mit dem Ursprung des Strahls und $t_{fern}$ werden Anfang, Ende und die Länge des Strahls berechnet. Mit Hilfe der Länge kann ermittelt werden um wie weit pro Iteration am Strahl entlang gegangen werden soll. Dadurch wird der Strahl nur bis zu seinem Austritt aus dem Würfel abgetastet. 

In einer for-Schleife wird jeder Strahl nun abgetastet. In jeder Iteration wird jeweils ein Punkt betrachtet. Der Punkt verschiebt sich entlang des Strahls um die zuvor berechnete Distanz.
Für jeden Punkt werden zuerst die uv-Koordinaten berechnet, da sonst nur ein viertel des MRT-Bildes dargestellt würde.
% WARUM???
Für die Koordinaten werden dann die jeweiliges Isowerte aus der 3D-Textur gelesen, die zuvor mit den MRT-Bildern befüllt wurde.

%---------------SAMPLE VOLUME-------------
Hierbei werden lediglich die uv-Koordinaten als Indices für die Textur verwendet. 
Der Isowert ist dabei im Alphakanal der Textur gespeichert. Da es sich nicht um eine Farbe sondern nur einen Grauwert handelt, können die anderen Farbkanäle der Textur mit der ?Magnitude? des jeweiligen Pixels befüllt werden. Darauf wird in der Sektion Transferfunktion genauer eingegangen.
% VERWEIS Transferfunktion
Der Isowert wird außerdem noch mit der Intensität multipliziert, die der Nutzer beeinflussen kann.
An dieser Stelle wird aber auch geprüft, ob der betreffende Punkt überhaupt zu sehen ist, oder aufgrund der verschiebbaren Schichten nicht sichtbar sein sollte. 
Dazu wird zuerst der aktuell betrachtete Punkt mit der Rotationsmatrix des Modells multipliziert.
% UBERPRÜFEN Warum?
Der Punkt wird dann mit den minimalen und maximalen x-, y-, und z-Werten der verschiebbaren Schicht verglichen. Das Ergebnis des Vergleichs wird dabei in einer Variable gespeichert. Ist der Punkt kleiner als das Minimum oder größer als das Maximum wird 0 gespeichert, ansonsten 1. 
Die beiden Werte werden anschließend mit dem Isowert multipliziert.Ist einer der Werte null, ist auch der ermittelte Isowert null, was im Alphakanal totale Transparenz bedeutet. 


%-----------------------------------------
Dies tritt ein, da der Isowert zunächst für alle Farbkanäle verwendet wird, um eine neue Farbe zu deklarieren. ?

An dieser Stelle wird über die Transferfunktion der entsprechende Farbwert aus der zugehörigen Textur gelesen. Dazu wird der Isowert als Index verwendet. Die Funktionsweise und Implementierung der Transferfunktion wird in der Sektion \ref{transfer} beschrieben.
% Bezug Transferfunktion!
Die Transferfunktion wird nur abgerufen, wenn der Isowert nicht 0 ist, da sonst die Transparenz überschrieben würde.
% Tranferfunktion bei index 0 auf 0000 setzen??
Ist die Farbe bekannt, wir der betrachtete Voxel illuminiert. Dies ist in der Sektion \ref{ilumination} beschrieben. 

Der Alphawert der so erhaltenen Farbe wird noch einmal halbiert, um die Darstellung semi-transparent erscheinen zu lassen.
% WARUM

Schließlich wird der erhaltene Farbwert mit den vorhergehenden verrechnet. Die Komposition erfolgt dabei von vorne nach hinten, da der Strahl in dieser Richtung abgetastet wird, wie folgt:
% Referenz GPU Gems

$\hat{C}_{i}=(1-\hat{A}_{i-1})C_{i}+\hat{C}_{i-1}$

$\hat{A}_{i}=(1-\hat{A}_{i-1})A_{i}+\hat{A}_{i-1}$

Wobei $\hat{C}_{i}$ die Farbe und $\hat{A}_{i}$ die Transparenz der Farbe des vordersten Voxels ist.
Wenn diese Farbe einen vorher definierten Schwellenwert überschreitet wird die Schleife abgebrochen. Der Schwellenwert kann vom Nutzer manipuliert werden und bestimmt die ?Helligkeit? der Darstellung.

Die Farbe wird schlussendlich noch auf einen Wert zwischen 0 und 1 festgesetzt und mit der Farbe der Maske verrechnet.

% MASKE
% PHONG
% ILLUMINATION 


- Transferfunktion: Erstellen einer Textur

- Illumination: Gradient vector = normal
- Gradient: 



\subsection{Transferfunktion}
\label{transfer}

- lookup

- erzeugen von transferfunktion

\subsection{Illumination}
\label{illumination}

Das Volumen wird mit mit dem Phong Beleuchtungsmodell illuminiert, wodurch es mehr Plastizität erhält. Das Modell wurde bereits in Kapitel \ref{grundlagen} detailliert beschrieben. Deshalb wird an dieser Stelle nur auch die Umsetzung im Shader eingegangen.
% Referenz? warum nicht blinn?
Wie beschrieben setzt sich das Modell aus drei Komponenten zusammen: Der ambienten und der diffusen Beleuchtung, sowie der spiegelnden Reflexion. Die Komponenten werden zusammen addiert, was in die endgültige Farbe resultiert. 
Die Koeffizienten werden im Shader durch Farben repräsentiert. Der diffuse Koeffizient ist dabei der vorher aus der Transferfunktion gelesene Farbwert. Um diesen nicht durch die ambiente Beleuchtung zu verfälschen, wird er mit einem konstanten Faktor multipliziert, um eine abgedunkelte Farbe zu erhalten, die als ambienter Wert verwendet wird. Die Reflextion wird weiß dargestellt.
Als Reflexionsexponent hat sich der Wert ?10? als am besten erwiesen.

Um die Reflexion berechnen zu können müssen außerdem der Lichtvektor und die normale bekannt sein.
Unterscheidung direktionales und Punktlicht?

\subsection{Gradientenberechnung}
\label{gradienten}