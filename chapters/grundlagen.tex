% Grundlagen & verwnadte Arbeiten

\chapter{Aktueller Stand der Technik}

%-------------------------------------------------------------
\section{MRT}												 %
%-------------------------------------------------------------

Die Abkürzung MRT steht für Magnetresonanztomographie. Das bildgebende Verfahren, das auch Kernspintomografie genannt wird, wird in der Radiologie verwendet, um Abbildungen innerer Organe zu erzeugen. Während der Durchführung einer MRT wird der Patient in ein MRT-System geschoben, das einer großen Röhre gleicht. Er sollte sich für die Dauer des MRTs möglichst wenig bewegen, um klare Bilder zu erhalten.
 	
\subsection{Verfahren}

Um die inneren Organe eines Patienten zu visualisieren, werden kleinste Teilchen seines Körpers in Bewegung versetzt, die gemessen werden kann. Im Falle einer MRT handelt es sich dabei um Wasserstoffprotonen. Diese haben eine Eigendrehung um sich selbst, den sogenannten Kernspin. Durch ihre positive Ladung, die durch den Kernspin in Bewegung ist, besitzen die Protonen weiterhin ein eigenes Magnetfeld, welches messbar ist. 
Während einer MRT wird mit einer Spule, die in dem MRT-System verbaut ist um den Körper des Patienten ein Magnetfeld erzeugt. Die Kernspin-Achsen der Wasserstoffprotonen richten sich an diesem aus. Anschließend wird in das Magnetfeld ein Hochfrequenzimpuls, die Larmorfrequenz eingestrahlt. Durch diesen Impuls findet eine Synchronisation der Protonen statt, wobei einige um 180° gedreht werden. Kurz danach laufen die Protonen wieder auseinander und richten sich wieder am Magnetfeld aus. 
Dadurch, dass alle Protonen in dieselbe Richtung zeigen (phasengleich sind), verstärken sie gegenseitig das Signal, dass sie abgeben. Das Signal wird schwächer, sobald sie wieder auseinander laufen (Dephasierung).
Die Zeit, die die Protonen brauchen, um sich wieder am Magnetfeld auszurichten wird als Relaxtionzeit bezeichnet. Dabei wird zwischen T1- und T2-Relaxtion unterschieden.
Die Relaxionszeit ist dabei abhängig von der Zusammensetzung des umgebenden Gewebes. Das gemessene MR-Signal, das durch diese beeinflusst wird, ist also für verschiedene Gewebearten verschieden stark.

\subsection{Datenverarbeitung}

Da die MRT ein dreidimensionales Objekt abbilden soll, werden mit dem eben beschriebenen Verfahren die Werte auch der XY-Ebene erfasst. Das Gehirn wird in Z-Richtung in einzelne Schichten unterteilt. Eine Schicht ist dabei gewissermaßen ein Bild mit X- und Y-Koordinaten, von dem viele hintereinander gehängt werden. Diese Unterteilung in Schichten wird erreicht, indem Z-Gradientenspulen verwendet werden, die das Magnetfeld inhomogen machen und entsprechend dem Z-Gradienten zu einer Seite hin abfallen lassen. So kann jeder Z-Schicht eine bestimmt Stärke im Magnetfeld zugewiesen werden. 
Es wird immer nur eine Schicht auf einmal gescannt und verarbeitet.

Die X- und Y-Werte einer Schicht repräsentieren allerdings nicht, wie bei einem Bild Koordinaten, die der Anordnung der jeweils betrachteten Punkte in der Welt entsprechen. Stattdessen bildet der X-Wert die Frequenz und der Y-Wert die Phase ab. Wie für den Z-Wert werden auch hier Gradienten gebildet und auf das Magnetfeld gelegt. Der X-Gradient verläuft von links nach rechts und sorgt dafür, dass die Larmorfrequenz in dieser Richtung zunimmt, sodass die jeder Punkt seine eigene Frequenz hat.Der Y-Gradient, der senkrecht verläuft, beeinflusst auf dieselbe Weise die Phasen einer Schicht. Er wird dabei nur kurz nach dem Einstrahlen des Hochfrequenz-Impulses eingeschaltet, wenn sich die Protonen bereits ausgerichtet haben.

Für jeden Punkt gibt es also eine Magnetfeldstärke (Z), eine Frequenz (X) und eine Phase (Y). Diese Werte aller Punkte werden in einer Matrix gespeichern, die K-Raum genannt wird. Die Matrix entspricht allerdings noch nicht der bildlichen Darstellung, die angestrebt wird, da die Werte eine andere Bedeutung haben. ? Deshalb werden sie mit Hilfe der Fouriertransformation in lesbare Bilddaten umgewandelt, die die entsprechenden Organe schichtenweise abbilden. 

\subsection{Abgrenzung CT}
Eine von der Durchführung ähnliche Methode zur Abbildung des Körperinneren, ist die Computer-Tomographie. Die Verfahren unterscheiden sich jedoch. Denn bei einer CT wird der Patient schichtenweise geröntgt. Dh. sein Körper wird mit Röntgenstrahlung beschossen, die je nach Gewebe, auf das sie treffen unterschiedlich stark abgeschwächt werden, was dann gemessen wird. Die so entstandenen "Querschnitte" des Körpers werden anschließend mit Hilfe eines Computers zu einem dreidimensionalen Bild zusammengesetzt. 

Die CT ist deutlich kürzer als eine MRT. Deshalb wird sie oft bei Notfällen verwendet. Allerdings wird der Patient dabei auch der Belastung von radioaktiver Strahlung ausgesetzt ist, die stärker ist als beim normalen Röntgen. Außerdem ist können Weichteile mit einer MRT besser darstellt werden. Sie eignet sich also mehr zur Untersuchung des Gehirns.


\subsection{Datenformate}

MRT-Bilder werden meist in Dateiformaten gespeichert, die in der Medizin üblich sind. Dazu gehören nifti oder DICOM, welches z.B. neben den Bildern auch Patientendaten speichert.
Allerdings 
16 bit int images, pvm ? Standart?

%------------------------------------------------------
\section{Volumendaten}							  %
%------------------------------------------------------

Das Volumen setzt sich aus mehreren dreidimensionalen Bildpunkten zusammen. Diese werden Voxel genannt, vergleichbar mit Pixeln in zweidimensionalen Bildern.

Voxel

3D Texturen

Isosurface

%------------------------------------------------------
\section{Volume Rendering}							  %
%------------------------------------------------------
%https://developer.nvidia.com/gpugems/GPUGems/gpugems_ch39.html
% Bezug zu Motivation
Wie in Kapitel \ref{motivation} erläutert, sollen die MRT-Bilder, die der Radiologe untersucht in mARt als dreidimensionales Volumen dargestellt werden. Volume Rendering bezeichnet die Darstellung eines dreidimensionalen Volumens, meist durch ein Skalarfeld repräsentiert, auf einer zweidimensionalen Bild. MRT-Bilder, die in Graustufen vorliegen bilden ein solches Skalarfeld. 
Dadurch, dass das Volumen aus Voxeln gebildet wird, gibt es keine Oberfläche, die das abzubildende Objekt beschreibt. Die Darstellung erfolgt deshalb anhand eines optischen Modells. Jedem Wert im Datensatz werden dazu optische Eigenschaften zugewiesen, im Allgemeinen Farbe und Opazität. Indem Strahlen durch das Volumen geschossen werden, werden diese Eigenschaften mit einander verrechnet, was schließlich zur Abbildung des gesamten Volumen führt. Diese Technik wird im Abschnitt \ref{rayCasting} genauer beschrieben.
Die optischen Eigenschaften eines Voxels sind abhängig von seinem Isowert, sowie der verwendeten Transferfunktion und dem Shading. 

Eine Transferfunktion besteht meistens aus einer Textur, aus der für jeden Isowert eine Farbe und Opazität gelesen werden kann. Auf diese Weise können Voxel, die einem bestimmten Bereich oder z.B. Gewebe angehören hervorgehoben oder ausgeblendet werden. Eine gut Transferfunktion zu implementieren ist sehr schwierig, da die korrekten Werte oft nur durch Ausprobieren gefunden werden und von dem jeweiligen Datensatz abhängen. 

Das Shading bestimmt unter anderem die Beleuchtung eines Voxels und kann dessen Eigenschaften somit zusätzlich zur Transferfunktion verändern. Hierbei können verschiedene Beleuchtungsmodelle verwendet werden.
Bei einer lokalen Beleuchtung der einzelnen Voxel kommt hierzu meist das Phong-Beleuchtungsmodell zum Einsatz. 

%-------
Phong
Normalen
Gradienten
etc...
?
%-----1

Wie bereist erwähnt, werden die einzelnen Werte entlang eines Strahls mit einander verrechnet. Dabei werden sie in der Regel auf einander addiert. Dieser Vorgang wird Komposition genannt. Dabei ist es von Relevanz, in welcher Reihenfolge die Werte durchlaufen werden. Wird das Volumen von vorne nach hinten durchlaufen, erfolgt die Komposition wie folgt:

$\hat{C}_{i}=(1-\hat{A}_{i-1})C_{i}+\hat{C}_{i-1}$

$\hat{A}_{i}=(1-\hat{A}_{i-1})A_{i}+\hat{A}_{i-1}$

Wobei $\hat{C}_{i}$ die Farbe und $\hat{A}_{i}$ die Transparenz der Farbe des vordersten Voxels ist.
Sollen die Werte von hinten nach vorne addiert werden, wird dies durch die folgende Formel beschrieben.

$\hat{C}_{i}=C_{i}+(1-\hat{A}_{i})\hat{C}_{i+1}$

$\hat{A}_{i}=A_{i}+(1-\hat{A}_{i})\hat{A}_{i+1}$
 
Skalarfeld 
Optical Model
Classification?
Lightning

Es gibt verschiedne Techniken, von denen die wichtigsten im Folgenden erläutert werden.

\subsection{Volumetrisches Ray-Casting}
\label{rayCasting}

\textbf{Ray-Casting} ist eine bekannte Rendering-Methode, die auch beim Rendern von 3D-Szenen zur Anwendung kommt die keine Volumendaten enthalten.
In diesem Fall werden von der Position der Kamera für jeden Pixel Strahlen in die Szene geschossen. Für jeden Strahl wird dann errechnet, ob dieser die Oberfläche eines der Objekte in der Szene schneidet. Ist dies der Fall, wird der betreffende Pixel in der Farbe des Objektes eingefärbt, wobei der verwendete Shader diese beeinflusst. Es wird außerdem berücksichtigt, welche Länge der Strahl zu dem Schnittpunkt hat, da nur der Eintrittspunkt relevant ist.
Weiterhin existieren die Techniken Ray-Tracing und Ray-Marching, die ebenfalls auf der Kollision zwischen Strahlen und Objekten beruhen.
Beim \textbf{Ray-Tracing} werden dabei neben dem ursprünglichen Strahl, der das Objelt trifft noch weitere berechnet, die durch die gesamte Szene laufen können um z.B. die Reflexion von Objekten aufeinander zu ermitteln. 
\textbf{Ray-Marching} ist eine schnellere Version des Ray-Castings. Hier wird der Schnittpunkt von Strahl und Oberfläche nicht genau kalkuliert. Stattdessen wird entlang des Strahls in kleiner werdenden Abständen jeweils ein einzelner Punkt betrachtet. Für diesen Punkt wird lediglich geprüft, ob er sich bereits innerhalb des Objektes befindet oder nicht. Der erste Punkt, auf den dies zutrifft wird als Schnittpunkt angesehen. Obwohl diese Berechnung etwas weniger genau ist, ist wie bereits gesagt schneller als Ray-Casting.

%Ray-Casting bzw. Ray-Marching wird auch zum Rendering von Volumen benutzt.
Bei beiden Verfahren muss allerdings die Oberfläche bzw. die Form des Objektes bekannt sein, um bestimmen zu können, wann der Strahl in sie eintritt. Dies ist bei Volumendaten in der Regel nicht gegeben, da nicht zur das Objekt selbst, sondern auch dessen Umgebung darin gespeichert sind. Weiterhin werden durch die eben beschriebenen Prozesse nur die Oberflächen der Objekte gerendert, nicht aber ihr Inneres.

Hier liegt der Unterschied zum Rendering von Volumen. In diesem Fall werden beide Verfahren quasi vereint. \textbf{Volumetrisches Ray-Casting} und volumetrisches Ray-Marching bezeichnen deshalb ein und dasselbe. Auch hier werden dabei Strahlen von der Kamera in die Szene geschossen. Um das Volumen zu rendern sind davon nur die Strahlen relevant, die das Volumen treffen. Und auch von diesen wird nur der Teil des Strahls berücksichtigt, der innerhalb der umgebenden Geometrie liegt. Für jeden Strahl Ein- und Austrittspunkt errechnet. Innerhalb der Geometrie werden jetzt entlang des Strahls in bestimmten Abständen Voxel betrachtet. Für jeden Voxel wird ein Farb- und Alphawert bestimmt und die Werte werden abhängig von der Komposition von hinten nach vorne oder von vorne nach hinten aufeinander addiert. Die Summen der Werte ergeben so eine dreidimensionale Darstellung des Volumens aus der Sicht der Kamera.

\subsection{Texture Based Volume Rendering}

Beim texturbasierten Rendering von Volumen wird eine dreidimensionale Darstellung erzeugt indem viele Texturen aufeinander geschichtet werden, die dann mit Alpha Blending übereinander geblendet werden. Dazu wird für jede Schicht ein Querschnitt durch das Volumen gemacht, auf den dann mit Hilfe von Texture Mapping die Textur gerendert wird.
Die Form und Ausrichtung der Querschnitte wird dadurch bestimmt auf welche Art die Volumendaten vorliegen. In Abbildung \ref{images/2D3DTex} wird der Unterschied verdeutlicht. Handelt es sich um zweidimensionale Texturen,  richten sich die Querschnitte am Volumen aus. Innerhalb eines Würfel sind sie also quadratisch. Damit das Volumen aus verschiedenen Blickwinkeln betrachtet werden kann, wird für jede der Hauptsichtachsen ein Stapel mit Texturen erstellt. Zwischen den Stapeln wird gewechselt, sobald sich der Betrachtungswinkel ändert. Dies ist in Abbildung \ref{images/textureBased} dargestellt.

Liegen die Daten als 3D-Texturen vor, verlaufen die Querschnitte entlang der Sichtachse. Dazu werden sogenannte Proxy Geometrien erzeugt, Polygone, die einen Querschnitt beschreiben. Auf diesen Proxy Geometrien wird dann die Textur erstellt, indem die entsprechenden Volumendaten abgefragt werden. Dadurch, dass die Querschnitte an der Sichtachse ausgerichtet sind, ist nur ein Texturstapel notwendig. 


https://dl.acm.org/citation.cfm?id=329138

\subsection{Shear-Warp}

Shear-Warp verfolgt den selben Ansatz wie das Ray-Casting. Anstatt die Strahlen allerdings von der tatsächlichen Kameraposition aus zu verschießen, werden die Strahlen orthogonal zu den Volumenschichten in das Volumen gefeuert. Auf diese Weise wird die Berechnung der Strahlen und die Komposition der Darstellung deutlich beschleunigt. 
Damit auch bei einem nicht orthogonalen Blickwinkel das Volumen korrekt dargestellt wird, wird vor dem Ray Casting eine vom Blickwinkel abhängige Scherung auf die Volumendaten angewandt. In Abbildung \ref{shearwarp} ist Dargestellt, wie die Scherung die einzelnen Schichten entsprechend des Blickwinkels verschiebt, um einen orthogonalen Einfallswinkel zu simulieren. 
Das durch das Ray Casting entstandene Bild wird zunächst in einen Buffer gerendered. Durch die Scherung ist das Bild zu diesem Zeitpunkt noch verzerrt. Deshalb wird es anschließend noch einmal transformiert, sodass eine korrekte Abbildung des Volumens auf die Bildschirmebene projiziert wird. In Abbildung \ref{shearwarpResult} ist zu sehen, wie das gerenderte Bild vor und nach der Transformation aussieht. 

Shear-Warp vereint damit die Bildqualität des Ray-Castings, ist aber dabei um einiges schneller.

%-------------------------------------------------------------
\section{Weitere Methoden zur dreidimensionalen Darstellung von MRT-Daten}		 %
%-------------------------------------------------------------
% Bsp
Wie in Kapitel \ref{motivation} beschrieben wurde, bietet eine 3D-Darstellung von MRT-Bildern einige Vorteile gegenüber einer Betrachtung der Daten in 2D. 
% Kann man das belegen?
Auf Grund dieser Tatsache gab es im Laufe der Zeit verschiedene Ansätze zur Umsetzung einer dreidimensionalen Darstellungsweise. 
Im Folgenden werden diese erläutert. 

% Vor- und Nachteile Tabelle?


\subsection{Marching Cubes}

\subsection{Voxel}

\subsection{Andere}
https://dl.acm.org/citation.cfm?id=3264776


%-------------------------------------------------------------
\section{Vergleich der Methoden im Überblick}											 %
%-------------------------------------------------------------

%-------------------------------------------------------------
\section{AR und VR}											 %
%-------------------------------------------------------------

%-------------------------------------------------------------
\section{Verarbeitung von MRT-Daten}						 %
%-------------------------------------------------------------
Obwohl die Darstellung von MRT-Bildern in direktem Zusammenhang mit ihrem Zweck steht, bleibt die Verarbeitung der Daten auch bei verschiedenen Darstellungsformen gleich. 

\subsection{Arbeit eines Radiologen}
Wie bereits in Kapitel \ref{motivation} oberflächlich erläutert wurde, besteht der Nutzen eines MRTs darin, dass der behandelnde Arzt einen Einblick in die betreffenden Organe (hier das Gehirn) erhält. Dazu studiert er die einzelnen Schichten des Gehirns und hält Ausschau nach Anomalien.
Im Fall von Schlaganfällen sind diese ...

Wurde eine entsprechender Bereich identifiziert, markiert der Arzt diesen auf jeder einzelnen Schicht. 
\subsubsection{Schlaganfälle auf einem MRT}

\subsection{Software in der Radiologie}
%Use Cases
%Bsp
Die Untersuchung der MRT-Daten werden digital durchgeführt. Um dem Arzt einen Einblick in den Datensatz zu geben, gibt es spezielle Software, die diesen darstellen kann und weiterhin relevante Funktionen bietet. Beispiele für solche Software sind die Programme


%-------------------------------------------------------------
\section{Interaktion in AR/VR}								 %
%-------------------------------------------------------------
\subsection{Integrierte Nutzereingaben}
\subsection{Leap Motion}
\subsection{Andere}







