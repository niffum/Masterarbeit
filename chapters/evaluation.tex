% Evaluation


\chapter{Evaluation}
\label{evaluation}

\section{Vergleich mit Anforderungen}

Zunächst werden in der folgenden Tabelle \ref{tab:evaluation} die implementierten Funktionen von mARt mit den in Kapitel \ref{anforderung} aufgestellten Anforderungen abgeglichen. Die Anforderung ist dabei über die entsprechende User Story referenziert. Dazu ist angegeben, inwiefern die Anforderung erfüllt wurde, so wie eventuelle Anmerkungen. 

\begin{longtable} {p{.125\textwidth}p{.225\textwidth}p{.60\textwidth}}
\toprule
User Story & Erfüllung & Anmerkungen \\
\toprule
U01 & Erfüllt & Die Darstellung der Daten ist zu großen Teilen abhängig von diesen selbst. Die Darstellung in AR ist abhängig von den Lichtverhältnissen.\\
\midrule 
U02 & Erfüllt & Die Betrachtung aus einem Winkel ist möglich.\\
\midrule 
U03 & Erfüllt & \\
\midrule 
U04 & Erfüllt & Die Anzahl der angezeigten Datensätze ist auf zwei beschränkt.\\
\midrule 
U05 & Erfüllt & Die Daten, aus denen der Nutzer wählen kann sind fest in das Programm integriert und es kann nur aus zwei Datensätzen gewählt werden.\\
\midrule
U06 & Erfüllt erfüllt & Maximal zwei Darstellungen können gleichzeitig manipuliert werden. \\
\midrule 
U07 & Erfüllt & Die zweidimensionale Darstellung kann auf einer Achse verändert werden. Die dreidimensionale auf drei.\\
\midrule 
U08 & Offen & \\
\midrule 
U09 & Erfüllt & \\
\midrule 
U10 & Teilweise erfüllt & Manipulationen außer Skalierung und Rotation werden übernommen, sofern der Wert in beiden Szenarien existiert.\\
\midrule 
U11  & Erfüllt & \\
\midrule
U12 & Erfüllt & Eine Manipulation der Maske würde dem Nutzer eine Anpassung erlauben, wurde jedoch nicht umgesetzt.\\
\midrule 
U13 & Erfüllt & \\
\midrule 
U14 & Erfüllt & \\
\midrule 
U15 & Erfüllt & Nur bei Anzeige eines Datensatzes möglich. Temporäte Manipulation.\\
\midrule 
U16 & Erfüllt & Die position wird nicht über mehrere Sitzungen hinweg gespeichert.\\
\midrule 
U17 & Offen & Nicht umsetzbar durch Abhängigkeit von einem PC.\\
\midrule 
U18 & Offen & Keine unterschiedlichen Sequenzen standen zur Verfügung.\\
\midrule 
U19 & Offen & Datensätze sind in Anwendung integriert.\\
\midrule 
U20 & Offen & Datensätze sind in Anwendung integriert.\\

\bottomrule
\caption{\label{tab:evaluation}Erfüllung der Anforderungen.}
\end{longtable}

\section{Nutzertest}

Um die Verwendung von mARt im Bereich der Schlaganfallbehandlung zu evaluieren wurde ein Nutzertest abgehalten. 
Dabei wurden sowohl die AR- als auch die VR-Anwendung von einem Neurologen getestet. 

Ergebnisse

\section{Auswertung}
% Was ist besser als vorher? Warum?