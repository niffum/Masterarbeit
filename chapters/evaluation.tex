% Evaluation


\chapter{Evaluation}
\label{evaluation}

\section{Vergleich mit Anforderungen}

Zunächst werden in der folgenden Tabelle \ref{tab:evaluation} die implementierten Funktionen von mARt mit den in Kapitel \ref{anforderung} aufgestellten Anforderungen abgeglichen. Die Anforderung ist dabei über die entsprechende User Story referenziert. Zusätzlich wird diese stichwortartig beschrieben, um den Inhalt der Story wiederzugeben. Dazu ist angegeben, inwiefern die Anforderung erfüllt wurde, so wie eventuelle Anmerkungen. 

\begin{longtable} {p{.125\textwidth}p{.2\textwidth}p{.07\textwidth}p{.5\textwidth}}
\toprule
User Story & Stichwort & Status & Anmerkungen \\
\toprule
U01 & Gut erkennbare Abbildung & Erfüllt & Die Darstellung der Daten ist zu großen Teilen abhängig von diesen selbst. Die Darstellung in AR ist abhängig von den Lichtverhältnissen.\\
\midrule 
U02 & 2D-Darstellung der Daten aus min. einem Blickwinkel & Erfüllt & Die Betrachtung aus einem Blickwinkel ist möglich.\\
\midrule 
U03 & 3D-Darstellung & Erfüllt & \\
\midrule 
U04 & Min. zwei Datensätze anzeigen & Erfüllt & Die Anzahl der angezeigten Datensätze ist auf zwei beschränkt.\\
\midrule 
U05 & Datensätze auswählen & Erfüllt & Die Daten, aus denen der Nutzer wählen kann sind fest in das Programm integriert und es kann nur aus zwei Datensätzen gewählt werden.\\
\midrule
U06 & optionale gleichgeschaltete Manipulation & Erfüllt erfüllt & Maximal zwei Darstellungen können gleichzeitig manipuliert werden. \\
\midrule 
U07 & Scrolling auf min. einer Achse & Erfüllt & Die zweidimensionale Darstellung kann auf einer Achse verändert werden. Die dreidimensionale auf drei.\\
\midrule 
U08 & Scrollen mit Scrollrad & Offen & \\
\midrule 
U09 & Wechsel von 2D- und 3D-Darstellung & Erfüllt & \\
\midrule 
U10 & Übertragung der Manipulationen & Teilweise erfüllt & Manipulationen außer Skalierung und Rotation werden übernommen, sofern der Wert in beiden Szenarien existiert.\\
\midrule 
U11  & Ein- und Ausblenden des markierten Bereichs & Erfüllt & \\
\midrule
U12 & Erkennbare Strukturen bei Einblendung & Erfüllt & Eine Manipulation der Maske würde dem Nutzer eine Anpassung erlauben, wurde jedoch nicht umgesetzt.\\
\midrule 
U13 & 3D-Darstellung drehen & Erfüllt & \\
\midrule 
U14 & Manipulation von Helligkeit und Kontrast & Erfüllt & \\
\midrule 
U15 & Skalierung der Darstellungen & Erfüllt & Nur bei Anzeige eines Datensatzes möglich. Temporäte Manipulation.\\
\midrule 
U16 & Positionierung der Darstellung & Erfüllt & Die position wird nicht über mehrere Sitzungen hinweg gespeichert.\\
\midrule 
U17 & Beibehalten der Position & Teilweise erfüllt & Die Positionierung wird für die Dauer einer Sitzung gespeichert und wir relativ zum Nutzer gesetzt.\\
\midrule 
U18 & Auswahl von Scans & Offen & Keine unterschiedlichen Sequenzen standen zur Verfügung.\\
\midrule 
U19 & Unterstützung des NIfTI-Formats & Offen & Datensätze sind in Anwendung integriert.\\
\midrule 
U20 & Unterstützung des DICOM-Formats & Offen & Datensätze sind in Anwendung integriert.\\

\bottomrule
\caption{\label{tab:evaluation}Erfüllung der Anforderungen.}
\end{longtable}

\section{Nutzertest}
\label{nazuertest}

Um die Verwendung von mARt im Bereich der Schlaganfallbehandlung zu evaluieren wurde ein Nutzertest abgehalten. 
Dabei wurden sowohl die AR- als auch die VR-Anwendung von einem Neurologen getestet. Anschließend wurde dieser befragt.

% VR und AR Vergleich
Obwohl die beiden Anwendungen von Funktionsumfang und Bedienung her gleich sind, wurde die Nutzung der VR-Anwendung als angenehmer beschrieben. 
Sie wurde als flüssiger und stabiler wahrgenommen. 
Dies ist vor allem auf die Verzögerung zurückzuführen, die entsteht, wenn die Inhalte vom Rechner auf die \textit{HoloLens} übertragen werden. Da sie virtuellen Hände sich nicht in Echtzeit mit den realen bewegen, wird das Nutzungserlebnis negativ beeinflusst. 
Die Verzögerung führte auch dazu, dass Interaktionen teilweise nicht wie erwartet abliefen. 
Ein weiterer Grund für fehlerhafte Interaktionen ist das eingeschränkte Blickfeld der \textit{HoloLens}. Wie bereits beschrieben wurde, werden Inhalte nur in diesem Blickfeld dargestellt. Der Bereich, in dem die \textit{Leap Motion} die Hände des Nutzers erfasst ist allerdings um einiges größer. Dadurch werden Gesten häufig auch dann registriert, wenn der Nutzer seine Hand außerhalb des eigenen Blickfeldes bewegt. Dies wird verstärkt, durch die Tatsache, dass die \textit{Leap Motion} mit einer Neigung nach unten auf der \textit{HoloLens} angebracht wird, wodurch sich ihr Erfassungsbereich ebenfalls verschiebt.
Manipulationen, die vermehrt nicht erwartungskonform durchgeführt wurden waren die Skalierung von Ansichten, sowie die Rotation der 3D-Darstellung.

% Schwierigkeiten in der Bedienung 
Die Schwierigkeiten bei der Bedienung waren teilweise ebenfalls in der VR-Anwendung festzustellen. Grundsätzlich war die Darstellung Reaktion des Programmes zwar stabiler, einzelne Interaktionen, wie das Verschieben von Schiebereglern funktionierten allerdings nicht immer entsprechend der Erwartung des Nutzers, sodass dieser mehrfach für eine Interaktion ansetzen musste. 
Diese Fälle ließen sich allerdings kaum verlässlich reproduzieren und sich wahrscheinlich auf die \textit{Leap Motion} Technologie zurückzuführen.

Ein großer Kritikpunkt beider HMDs war außerdem der mangelnde Tragekomfort. Durch den Aufbau und das Gewicht der Brillen war das Tragen dieser für den Tester überaus unangenehm. 

Weiterhin wurden Anmerkungen zu sinnvollen bzw. notwendigen Erweiterungen der Anwendungen gemacht.
Beispielsweise könnte dem Nutzer Rückmeldung gegeben werden, wenn eine Darstellung mit beiden Händen gegriffen wurde, sodass es ihm leichter fällt eine Skalierung nachzuvollziehen.
Die in Kapitel \ref{konzept} beschriebene Gammakorrektur, wurde für die Untersuchung der Daten nicht als ausreichend angesehen. Stattdessen müssten Helligkeit und Kontrast separat voneinander manipulierbar sein, da Datensätze in unterschiedlichen Wertebereichen vorliegen können. 
Schließlich wurde durch den Tester angemerkt, dass obwohl für diesen Prototyp nur das Scrollen in einer Richtung für die 2D-Darstellung gefordert war, die Betrachtung der Daten aus verschiedenen Blickwinkeln für einen realen Einsatz unabdingbar sei. Dies gilt für mehrere Manipulationsmöglichkeiten, wie die Beeinflussung von Farbe und Opazität der gekennzeichneten Schlaganfallbereichs.

Über die vorhandenen Bedienmöglichkeiten sagte der Tester aus, dass sie leicht verständlich und schnell erlernbar waren, was teilweise dem eingeschränkten Umfang der Anwendung zuzuschreiben ist.
Besonders positiv aufgefallen ist die Platzierung von Bedienelementen im ausklappbaren Handmenü, welches bestimmte Optionen verbirgt bis der Nutzer beschließt den Fokus auf diese zu legen. Dabei wurde angemerkt, dass das Menü mit mehr Optionen ausgebaut werden könnte. 

Die Anwendung wurde im Allgemeinen als positiv aufgenommen. 
Die Darstellung der Daten wurde als deutlich genug eingeschätzt, um eine Untersuchung und Diagnose durchzuführen. 
Vor allem die dreidimensionale Darstellung der MRT-Daten im Raum wurde als sehr anschaulich beschrieben. Der Tester hatte den Eindruck, dass das Verständnis der angezeigten Daten, insbesondere von Lage und Größe des vom Schlaganfall betroffenen Bereichs, im Vergleich zur Darstellung auf einem Bildschirm verbessert wird. Die direkte Interaktion mit der Darstellung, wie die Rotation durch Greifen verstärkt diesen Effekt zusätzlich. Dadurch kann die aktuelle Situation auch Personen ohne Fachwissen begreiflich vermittelt werden. 
Die Verwendung von mARt beschrieb der Tester weiterhin als unterhaltsam und interessant.

\todo{Weglassen?}
Es wurde allerdings auch betont, dass ein Einsatz von mARt im derzeitigen Zustand nicht realistisch ist und dass es die Funktion von derzeitigen Viewern nicht ersetzten kann. Gleichzeitig wurde aber das Potenzial der Anwendung hervorgehoben sowie die Vielzahl an Anwendungsfällen für die eine AR-Anwendung, wie mARt sinnvoll wäre. 

\section{Auswertung}
% Was ist besser als vorher? Warum?