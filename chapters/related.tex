% Related works

-> Zusammenführen mit Grundlagen
\chapter{Verwandte Arbeiten und Technologien}
% verwendete sofware bei radiologen
% unterscheidung ct und MRT
% verarbeitungsprozess von mRT: Veröffentlichungen-> Vince
% 
\label{related}

\section{AR in der Medizin}
% OPs
% Nutzen und Interaktion

\section{Visualiserung von MRT-Daten in AR/VR}

\cite{SORENSEN2001193}
- Erstellen eines Meshes anhand der MRT Daten
- Selbst entwickelte VR Umgebung, die Mesh darstellt und Interaktion erlaubt
- beschränkt sich auf Herz
- 2001

\cite{PMID:12635021}
- selbstgebautes AR-Headset
- markergestützt
- stellt MRT-Bilder auf Kopfatrappe dar
- soll Interventionen unterstützen
- 2003

\cite{Watts:2017:PAR:3139131.3141198}
- Projektion von 3D-MRT Bildern auf Patient
- Volume Rendering
- Aufbau der AR-Hardware steht im Vordergrund

\cite{Ghoshal:2017:STV:3173519.3173527}
-

\section{MRT Volume Rendering}

\section{Interaktion in AR}
% Leap Motion?
