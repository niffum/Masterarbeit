% Konzept

\chapter{Konzept}
\label{konzept}

Im folgenden werden Konzepte diskutiert, um die in Kapitel \ref{anforderung} herausgearbeiteten Anforderungen zu erfüllen.

\section{3D Darstellung}

- Da das innere der 3D Darstellung erkennbar sein soll, muss eine semi-transparente Darstellung erzeugt werden, die sowohl die Form des Gehirns abbildet, als auch die innere Struktur
- Relevant sind nur die Pixel, die das Gehirn darstellen. Der Bereich darum (Schädel und Hintergrund) muss gefiltert werden. Das gewünschte Ergebniss wäre, wenn das Gehirn frei im Raum "schwebt" (Bei Ray Casting noch mal erwähnen?)

- Daten liegen dreidimensional in Schichten vor
- verschiedene Möglichkeiten zur 3D Visualisierung
- - Voxel
- - Volume Rendering (Ray Casting)
- - Marching Cubes
- - ...?

- Ist es notwendig bzw. nützlich ein  Mesh zu generieren?
	- Gut für Darstellung der Gehirnform. Relevant ist aber vor allem das "Innere" des Gehirns, da sich dort der markierte Bereich befindet. 
	- Marching Cubes eignen nich nicht, um innere Struktur Darzustellen. Eine Transparente Ansicht würde das Modell als innen "gleich" darstellen
	- Das Mesh müsste kontinuierlich angepasst werden, um jeweils andere Schichten beim scrollen anzuzeigen.
	
	- Voxel könnten auch das Innere des Gehirns darstellen. Um ein deutliches und hochaufgelöstes Modell zu erhalten währen allerdings viele Voxel notwendig. Performance?
	- Referenzen
	
	- Im Fokus der Darstellung, soll das Innere des Gehirns stehen. Ein Mesh, dass die äußere Form beschreibt ist also nicht sinnvoll, zumal keine der Anforderungen Funktionen beschreibt, für die ein Mesh nötig wäre (z.B. Kollsion (Unity))
	
Wie in dem Kapitel \ref{related} beschrieben, gibt es bereits viele Lösungsansätze zur 3D Darstellung von MRT-Bildern. Obwohl verschiedene Methodne zum Einsatz kommen, ist der am weitesten verbreitete Ansatz der des Volume Rendering.
Dies hat die folgenden Gründe:
- 
- ...
	
Diese Vorgehensweise ist somit am besten für die Implementierung der Anwendung geeignet.
Diese wird im Kapitel \ref{implementierung} genauer beschrieben.

Anforderungen an Shader??


\section{Darstellung des gekennzeichneten Bereichs}


U02 U03  


\section{Endgerät}

- Hololens oder HTC Vive?
- Leistung
- Interaktionsmöglichkeiten
- Tragekompfort
- ...
% Bezug Hololens2?

\section{Interaktion/ UI } 

% 2D 
% UX steht im Vordergrung -> begründen warum besser als vorher!

Scrollen durch 2d Bilder:
kein Mausrad-> slide vor und zurück abhängig von armlänge
slide hochrunter recht links, schwierig, ungenau

Rad wäre gut, kleines Rad (Lautstärkeregler) ideal
aber handgelenkbewegung schwer zu tracken
deshalb: Ring, den man anfasen und drehen kann, Geräusche untermalen auswahl (Feeback)

scrollwn wird oft als swipr bewegung implementiert: ungenau zu stoppen

http://blog.leapmotion.com/designing-cat-explorer/

Manipulation: Helligkeit kontrast, Größe, Verschieben, schichten scrollen, Maske?, ...

Der Use Case (REFERENZ Vergleich) erfordert es weiterhin, dass man Bilder im direkten Vergleich nebeneinander betrachten kann. Dies gilt sowohl für die zwei- als auch dreidimensionale Darstellung der Daten. Die UseCases (REFERENZ Liste) und (REFERENZ synchon) erfordern außerdem eine Möglichkeit für den Nutzen aus den vorhandenen Daten einzelne auszuwählen , die angezeigt werden sollen, sowie die Daten jeweils gleichzeitig oder einzeln zu manipulieren. 
Die Auswahl der Datensätze, sowie die Wahl ob diese synchron manipuliert werden sollen oder nicht, sind beide nicht teil der direkten Manipulation des Bildes. Sie müssen deshalb nicht in dessen unmittelbaren Umfeld stehen und der Nutzer kann sich auf sie konzentrieren. 
Die Bedienung der entsprechenden Interaktionselemente sollte trotzdem intuitiv und schnell umzusetzten sein. 
Diese beiden optionen werden deshalb in einem Menü vereint. Dieses wird an der linken Hand des Nutzers verankert. Auf diese Weise kann es immer schnell erreicht werden und wird nicht unbeabsichtigt aus den Augen verloren. Die Interaktionselemente nutzen außerdem die Möglichkeiten der VR aus, da sie i auf einem Bildschirm nicht umsetzbar wären. 
Damit das Menü nicht wären der Verarbeitung der Bilder stört, wird es nur dann eingeblendet, wenn der Nutzer seine Handfläche Richtung Kamera dreht und diese somit ansieht. Motion Leap bietet bereits Beispiel Code, der dies umsetzt. ?
Eine besondere Herausforderung bei der Konzeption des Menüs ist es, dieses möglichst klein zu halten. Das Erfassungsfeld der Motion Leap Kamera, in dem die Hände erkannt werden ist beschränkt. Deshalb kann es bei einer Interaktionfläche, die viel Raum davon einnimmt dazu kommen, dass der Nutzer versehentlich Knöpfe betätigt. 
Um den Umfang des Menüs in benutzbaren Dimensionen zu halten wurde es auf drei Knäpfe reduziert. 
Die Liste der verfügbaren Datensätze kann darin allerdings nicht untergebracht werden. Deshalb kann sie bei Bedarf über einen der Knöpfe aus geklappt werden. 
Indem der Nutzer einen Datensatz auswählt, wird dieser auf der Bildfläche angezeigt. Die Auswahl ist auf zwei Datensätze beschränkt. Zu viele gleichzeitig dargestellte Bilder würden unübersichtlich wirken und im die Motivation hinter dem Use Case ist der direkte Vergleich zweier Bilder. Außerdem stellen mehr Bilder auch mehr Möglichkeiten dar diese in verschiedenen Kombinationen zu manipulieren. Dies hätte die Anwendung unnötig verkompliziert. 
Stattdessen wird die Synchronisierung der Manipulation über einen weiteren Knopf im Handmenü gesteuert. Dieser ist nur aktiv, wenn tatsächlich zwei Bilder angezeigt werden. Dann funktioniert er als Schalter. Indem der Nutzer in betätigt wird jeweils nur eine Benutzeroberfläche neben beiden Bildern angezeigt oder sie wird dupliziert und für jede Darstellung eingeblendet. 




% auf probleme bei der visualisierung auf Hololens hinweisen, VR App 

\subsection{Interaktion AR}

\subsection{Interaktion VR}
% Leap motion

U06  Als Nutzer möchte ich durch das 3D-Modelll scrollen können, um es vergleichbar mit der 2D Darstellung zu verwenden.

U07  Als Nutzer möchte ich die MRT-Darstellungen frei im Raum bewegen können, um sie meiner Position im Raum anzupassen.

U08 Als Nutzer möchte ich die MRT-Darstellungen skalieren können, um die Darstellung gut zu erkennen.

U09  Als Nutzer möchte ich die MRT-Darstellungen drehen können, um den besten Blickwinkel auf den für mich relevanten Bereich zu bekommen. 

U10  Als Nutzer will ich, dass die Darstellung nicht von anderen Elementen verdeckt wird, damit ich sie uneingeschränkt sehen kann. 

U11  Als Nutzer möchte ich mit einem Scrollrad durch die Darstellung scrollen können, um genaue Kontrolle darüber zu haben, welche Schichten angezeigt werden.

U12  Als Nutzer möchte ich alle vorhandenen MRT-Sequenzen sehen und zwischen ihnen wählen können, damit ich alle notwendigen Informationen zu dem Scan nutzen kann.  

U14  Als Nutzer möchte ich aus aus verschiedenen Darstellungen wählen können, die nebeneinander angezeigt werden, um direkte Vergleiche zwischen diesen ziehen zu können.


\section{Unterstützung von Dateiformaten} 

Die User Story U13 verlangt nach einer Unterstützung des nifti-Dateiformats.

nifti-Dateien, sind Bilddateien, die oft in der Medizin verwendet werden. Sie speichern Bildsequenzen und eigenen sich somit für MRT-Bilder. 
% Referenz nifti
Ein weiteres für diesen Zweck weitverbreitetes Dateiformat ist DICOM. ...
% Referenz DICOM

Es ist weiterhin sinnvoll gängigere Dateiformate, wie JPEG oder PNG zu unterstützen. 

Wie in \ref{anforderung} beschrieben, soll es sich bei mARt in erster Linie um einen Prototyp handeln. 
Der Datensatz, der dargestellt werden soll ist gering. Um die Komplexität und den Umfang der Anwendung möglichst gering zu halten kann darauf verzichtet werden, dem Nutzer die Möglichkeit zu geben, aus verschiedenen Dateitypen oder Datensätzen zu wählen. Es ist ausreichend, wenn der entsprechende Datensatz vor dem Build gewählt wird.

Da die Darstellung und interaktion eines Datensatzes also im Vordergrund stehhen soll, ist es sinnvoll, die Dateien in ein Format umzuwandeln, das einfacher zu verarbeiten ist (JPEG, PNG). Hierzu kann ein externes Tool verwendet werden (ImageJ Bezug in Implementierung).
