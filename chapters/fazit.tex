% Fazit

\chapter{Zusammenfassung und Fazit}

\section{Zusammenfassung}

Im Rahmen dieser Arbeit wurde untersucht, inwiefern eine interactive dreidimensionale Darstellung von MRT-Daten in einer AR-Anwendung einen Mehrwert für Neurologen im Bereich der Behandlung von Schlaganfällen bietet. 
Hierzu wurden zunächst technische Hintergründe zu MRT-Daten, den Möglichkeiten der 3D-Darstellung von diesen sowie AR- und VR-Technologien erläutert. Sowie damit in Zusammenhang stehende Arbeiten genannt. Einer der Schwerpunkte dieses Abschnitts war die Vorstellung verschiedener Volume Rendering Techniken.

Als Grundlage zur Konzeption der AR-Anwendung wurden dann Anforderungen aufgestellt, die durch Interviews mit einem Ratiologen der Charitßé Berlin erarbeitet wurden. Diese wurden als User Stories formuliert.

Anschließend wurde diskutiert, wie die einzelnen Funktionalitäten von mARt umgesetzt werden können. Hierbei wurden sowohl Aspekte der Implementierung als auch der Gestaltung von Bedienelementen und Aussehen der Anwendung berücksichtigt. Unter anderem wurde aus den zuvor beschriebenen Techniken die geeignetsten ausgewählt. 

Das beschriebene Konzept wurde implementiert und einzelne relevante Punkte der Umsetzung wurden erläutert. Die Ergebnisse der Implementierung wurden präsentiert.

Schließlich wurde die Anwendung mit den zuvor aufgestellten Anforderungen abgeglichen und von Neurologen getestet. 
Es konnte gezeigt werden, dass ...

\section{Ausblick}

...
Im Folgenden weiterführerende Entwicklungen von mARt erläutert.

\subsection{Implementierung von offene  Anforderungen}

Wie im Kapitel \ref{evaluation} dargelegt wurde, konnte nicht alle anfangs gestellten Anforderungen im Rahmen dieser Arbeit umgesetzt werden. 

WorldAnchor Hololens
Multiuser
Datensätze
...

\subsection{Verbesserte 3D Visualisierung}
Transferfunktionen
Testen von anderen verfahren
Globale illumination

\subsection{Übertragen der Anwendung auf andere Hardware}
Hololens2, Magic Leap

\section{Fazit}
