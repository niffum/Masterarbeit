% Fazit

\chapter{Zusammenfassung und Fazit}
\label{fazit}

\section{Zusammenfassung}

Im Rahmen dieser Arbeit wurde untersucht, inwiefern eine interaktive dreidimensionale Darstellung von MRT-Daten in einer AR-Anwendung einen Mehrwert für Neurologen im Bereich der Behandlung von Schlaganfällen bietet. 
Hierzu wurden zunächst technische Hintergründe zu MRT-Daten, den Möglichkeiten der 3D-Darstellung von diesen sowie AR- und VR-Technologien erläutert. Weiterhin wurden damit in Zusammenhang stehende Arbeiten genannt. Einer der Schwerpunkte dieses Abschnitts war die Vorstellung verschiedener Volume Rendering Techniken.

Als Grundlage zur Konzeption der AR-Anwendung wurden dann Anforderungen aufgestellt, die durch Interviews mit einem Neurologen der Charité Berlin erarbeitet wurden. Diese wurden als User Stories formuliert.

Anschließend wurde diskutiert, wie die einzelnen Funktionalitäten von mARt umgesetzt werden können. Hierbei wurden sowohl Aspekte der Implementierung als auch der Gestaltung von Bedienelementen und Aussehen der Anwendung berücksichtigt. Unter anderem wurden aus den zuvor beschriebenen Techniken die geeignetsten ausgewählt. 

Das beschriebene Konzept wurde implementiert und einzelne relevante Punkte der Umsetzung wurden erläutert. Die Ergebnisse der Implementierung wurden präsentiert.

Schließlich wurde die Anwendung mit den zuvor aufgestellten Anforderungen abgeglichen und von einem Neurologen getestet. 
Es konnte gezeigt werden, dass die Darstellung der dreidimensionalen Daten im Raum zu einem besseren Verständnis der Situation führt. Die direkte Interaktion mit der Darstellung trägt dazu bei. 

\section{Ausblick}

Wie mehrfach beschrieben wurde, handelt es sich bei mARt primär um einen Prototyp, der die Vorteile des Einsatzes einer AR-Anwendung zur Darstellung von MRT-Daten untersucht. Dementsprechend existieren viele Möglichkeiten das Programm in verschieden Richtungen zu erweitern.
Im Folgenden werden einige mögliche weiterführerende Entwicklungen von mARt erläutert.

\subsection{Implementierung offener  Anforderungen}

Wie im Kapitel \ref{evaluation} dargelegt wurde, konnte nicht alle anfangs gestellten Anforderungen im Rahmen dieser Arbeit umgesetzt werden. Um mARt für den Einsatz im Arbeitsumfeld nutzbar zu machen sollten diese erfüllt sein.

%U08
In User Story \textbf{U08} wird ein Scrollrad für den Wechsel zwischen den Schichten gefordert. Da die Steuerung von mARt im Rahmen dieser Arbeit allerdings durch Gesten erfolgt, wurde sie nicht erfüllt. In zukünftigen Konzepten lässt sich die Gestensteuerung allerdings möglicherweise mit anderen Bedienelementen kombinieren, beispielsweise einem Scrollrad, das multifunktional verwendet wird. Dies ist allerdings abhängig von dem angestrebten Anwendungsfall.

%U10 (teilweise)
Weiterhin wurde durch Story \textbf{U10} ein Wechsel zwischen 2D- und 3D-Darstellung beschrieben, bei dem die vom Nutzer getätigten Manipulationen in das jeweils andere Szenario übertragen werden. Diese Anforderung wurde nur teilweise erfüllt. Wie in Kapitel \ref{konzept} erläutert wurde, sind manche Manipulationen, wie die Skalierung nur temporär. In einem veränderten Konzept einer AR-Anwendung wie mARt, können sich Möglichkeiten ergeben alle Manipulationen zu übertragen oder sogar die Darstellungen synchron anzuzeigen.

%U17
Die in Story \textbf{U17} geforderte dauerhafte Positionierung im Raum wurde ebenfalls nur teilweise umgesetzt. Die Darstellung kann im Raum platziert werden. Allerdings behält sie diese Position nicht über mehrere Sitzungen hinweg. 
Weiterhin ist ihre Position relativ zur Position des Nutzers. Für den Prototyp ist dies ausreichend. Die User Story hat ihren Ursprung allerdings in der Idee einer Multiuser-Anwendung. Dabei können mehrere Nutzer die selbe Darstellung betrachten, die sich für alle an der selben Position im Raum befindet. Dieses Konzept wurde in Kapitel \ref{konzept} erläutert. Vor allem für Einsätze demonstrativer Art, wie Lernzwecke oder die Kommunikation mit Patienten wäre diese Funktion von Nutzen.

%U18
Eine weitere unerfüllte User Story ist \textit{U18}, die beschreibt, dass möglichst alle Sequenzen eines Scans eines Patienten wählbar sein sollen. Für diese Arbeit wurden nur eine begrenzte Anzahl an Datensätzen zur Verfügung gestellt. Bei der Untersuchung eines möglichen Schlaganfalls, kommen allerdings oft verschiedene Scans und Gewichtungen zum Einsatz, wie in Abschnitt \ref{mrt} beschrieben wurde. Sollten mehrere Datensätze zu einem Patienten vorliegen, ist es sinnvoll aus diesen wählen zu können. Dies steht allerdings in Zusammenhang mit der Implementierung einer generellen Auswahl von Daten, aus denen der Nutzer wählen kann, wie sie bereits dargelegt wurde.

%U19-20
Wie ebenfalls in Kapitel \ref{konzept} erläutert wurde, unterstützt der Prototyp nicht die Darstellung von vom Nutzer gewählten Dateien. Dementsprechend werden auch keine Dateien im DICOM- oder NIfTI-Format unterstützt. Für die Nutzung von Neurologen oder auch erweiterte Tests ist diese Funktion allerdings notwendig. Dazu müsste im Konzept der Anwendung eine Möglichkeit gefunden werden dem Nutzer die Auswahl von Daten zu erlauben und diese in das Programm zu übertragen. Dies setzt unter anderem voraus, dass ein Parser integriert wird, der die Daten aus besagten Formaten in einen für \textit{Unity} lesbaren Typ umwandelt. 

Hinzu kommen die Erweiterungen, die in Abschnitt \ref{nutzertest} beschrieben wurden. 
Abhängig von den konkreten Nutzungsszenarien werden sich außerdem weitere Anforderungen ergeben, um die mARt erweitert bzw. an die es angepasst werden muss. 
Es wäre wünschenswert, dass diese sowie zukünftige ähnliche Arbeiten einem Test mit einer größeren Anzahl an Testern unterzogen wird, da dies noch mehr Aufschluss über die Bedürfnisse verschiedener Nutzer und die Möglichkeiten einer AR-Anwendung liefert.

\subsection{Verbesserte 3D Visualisierung}

Die Ergebnisse des Volume Rendering der MRT-Daten wurde in Abschnitt \ref{ergebnisse} demonstriert. Der Nutzertest hat gezeigt, dass die Darstellungen von ausreichender Qualität sind, um eine Diagnose durchzuführen und vor allem den Zustand des Gehirns nach einen Schlaganfall zu vermitteln. 
Allerdings stellt das Volume Rendering für sich einen eigenen Forschungsbereich dar, der im Rahmen dieser Arbeit nicht abgedeckt werden konnte. Darin existieren viele Maßnahmen, um die Qualität der Darstellung weiter zu verbessern. Einige davon werden an dieser Stelle genannt. 

Zum einen wurde als Rendering Verfahren das Volume RayCasting gewählt, da es sehr gute Ergebnisse liefert. Je nachdem in welche Richtung sich zukünftige Arbeiten entwickeln, ist es allerdings sinnvoll weitere Verfahren zu testen, die beispielsweise weniger rechenintensiv sind. 
Weiterhin wurde zwar eine Transferfunktion verwendet, um das Rauschen der Bilder zu beschränken. Allerdings ist diese nicht für jeden Datensatz optimal und es wird lediglich ihr Alphawert genutzt. 
Idealerweise sollte dem Nutzer die Möglichkeit gegeben werden die Transferfunktion zu beeinflussen, wie es in Kapitel \ref{konzept} bereits erläutert wurde. Dies würde die Darstellungsqualität erhöhen und dem Nutzer zu einem Rendering nach seinen Ansprüchen verhelfen. 
Um das Rendering plastischer und realer erscheinen zu lassen kann außerdem ein globales Beleuchtungsmodell implementiert werden, wie es in Kapitel \ref{grundlagen} beschrieben ist. 
Da die Qualität des Renderings von den Bilddaten abhängt, können diese schließlich bereinigt bzw. aufgearbeitet werden. Wie in Abschnitt \ref{ergebnisse} dargelegt wurde, besteht in den gegebenen Datensätzen ein hohes Maß an Rauschen. Dieses könnte auf verschiedenen Wegen herausgefiltert werden, bevor die Bilder gerendert werden. Zudem könnte eine zusätzliche Interpolation vorgenommen werden, die die großen Abstände zwischen den einzelnen Schichten besser überbrückt als es bisher der Fall ist. Damit könnten Bildartefakte reduziert werden. 
Für die Verbesserung der MRT-Daten könnte beispielsweise maschinelles Lernen eingesetzt werden.

\subsection{Übertragen der Anwendung auf andere Hardware}
\label{hololens2Fazit}

Im Rahmen dieser Arbeit wurde mARt als AR-Anwendung nur auf der \textit{HoloLens} getestet. 
Die Verwendung der Anwendung zusammen mit anderen AR-Systemen, könnte noch mehr Aufschluss über die Möglichkeiten von AR-Systemen im medizinischen Bereich geben. 
mARt könnte dabei auf vergleichbaren AR-Systemen, wie der \textit{Magic Leap} von \cite{magicLeap} getestet werden.
Da die Entwicklung solcher Systeme momentan weiter fortschreitet, kommen allerdings auch noch nicht erschienene Geräte in Betracht.

Wie im Kapitel \ref{evaluation} beschreiben wurde, ist die Interaktion mit mARt durch das geringe Sichtfeld der \textit{HoloLens} erschwert. Die Inhalte werden abgeschnitten und die virtuellen Hände nehmen viel Platz ein. 
Dieses und andere Probleme, die bei der Verwendung von mARt auf der \textit{HoloLens} festgestellt wurden, könnten eventuell durch die Verwendung auf der \textit{HoloLens 2} behoben werden. 
Das Gerät wurde im Februar 2019 vorgestellt und soll voraussichtlich im April 2019 erscheinen. 
Die neue Version der \textit{HoloLens} besitzt ein mehr als doppelt so großes Sichtfeld, wie ihr Vorgänger (vgl. \cite{hololens2}), was dem eben genannten Problem entgegen wirkt. 
Weiterhin verfügt die \textit{HoloLens 2} über eine eingebaute Hand-Tracking-Funktion (vgl. \cite{hololens2}). Diese könnte an Stelle der \textit{Leap Motion} zum Erfassen der Hände des Nutzers verwendet werden. Dadurch entstehen zwei Vorteile gegenüber der aktuellen Anwendung. Zum Einen ist anzunehmen, dass die virtuellen Hände nicht länger angezeigt werden müssen. Dies vereinfacht die Benutzung und lässt mehr Raum im Sichtfeld für tatsächliche Inhalte. Zum Anderen, könnte die Anwendung für die \textit{HoloLens 2} bereitgestellt und somit kabellos verwendet werden, wodurch die Verzögerung der Übertragung auf das Gerät behoben und das Nutzungserlebnis verbessert würden. 

Weiterhin ist davon auszugehen, dass die \textit{HoloLens 2} über eine höhere Rechenleistung verfügt, als ihr Vorgänger. Dies könnte neben der eben genannten Optimierung des Renderingverfahrens eine flüssige Darstellung der 3D-MRT-Daten ermöglichen.

Eine Übertragung der Anwendung auf die \textit{HoloLens 2} könnte die Benutzbarkeit der Anwendung demnach stark erhöhen.

\subsection{Integration anderer Technologien}

Im vorherigen Abschnitt wurde bereits der Einsatz von maschinellem Lernen zur Verbesserung der Bildqualität angeführt. Diese Technologie könnte weiterhin zur Diagnose von Schlaganfällen oder zur Prognose ihrer Entwicklung verwendet werden. Funktionen dieser Art könnten in Anwendungen wie mARt  integriert werden.
\todo{Paper? P}

\section{Fazit}

Die Motivation hinter dieser Arbeit war es zu untersuchen, inwiefern eine interaktive AR-Anwendung die Untersuchung von volumetrischen MRT-Daten verbessern kann. Obwohl der dafür entworfene Prototyp längst nicht den Umfang eines MRT-Viewers abdeckt, konnten Schlüsse über die Möglichkeiten einer solchen Anwendung gezogen werden.
Der Nutzertest hat gezeigt, dass mARt zu einem besseren Verständnis der Situation führen kann. Dies wird durch die dreidimensionale Darstellung im Raum sowie die direkte Interaktion mit den Daten erreicht.
Eine AR-Anwendung dieser Art besitzt enormes Potenzial Nutzer in verschiedenen Anwendungsfällen zu unterstützen. Dazu gehört die Arbeit von Neurologen aber auch anderen Ärzten, sowie das Studium von Medizinstudenten und die Kommunikation mit Patienten. 

Bis AR-Anwendungen wie mARt Alltagsrealität werden bedarf es allerdings noch weiterer Entwicklungen.
Der derzeitige Stand der AR-Technologie stellt dabei das größte Hindernis dar. Wie in der Arbeit gezeigt werden konnte, ist die Darstellung von Volumendaten ein weitreichend erforschtes Gebiet. Um eine solche Darstellung über ein HMD in einer interaktiven Anwendung zur Verfügung zu stellen, muss dieses drei Voraussetzungen erfüllen:

\begin{itemize}
\item Die Leistung muss ausreichen, um hochwertige Renderings durchzuführen.
\item Es müssen intuitive und bequeme Interaktionsmöglichkeiten verfügbar sein.
\item Der Tragekomfort sollte eine Benutzung über einen längeren Zeitraum ermöglichen.
\end{itemize}

All dies trifft momentan nicht auf AR-HMDs wie die \textit{HoloLens} zu. Es ist wahrscheinlich, dass zukünftige Systeme der ständig voranschreitenden Technologie diese Eigenschaften mitbringen werden. 
Bis dahin gilt es zu untersuchen, was die notwendigen Interaktionen und relevantes Elemente für die jeweiligen Anwendungsfälle sind und wie diese sinnvoll in AR realisiert werden können, damit eine stabile und einsetzbare Anwendung entstehen kann, die das dargelegte Potenzial ausschöpft.
mARt ist somit der erste Schritt einer Entwicklung, um die Arbeit und Zusammenarbeit von Ärzten, Studenten und Patienten zu bereichern.
