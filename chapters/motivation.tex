% Motivation

\chapter{Einleitung}

Der Einsatz von MRTs ermöglicht es Ärzten einen Einblick in das innere des menschlichen Körpers zu erlangen, ohne diesen dabei zu verletzten. Sie erhalten Bilder innerer Organe, anhand deren sich dessen Aufbau und Funktionalität beobachten lassen, aber auch mögliche Fehlfunktionen oder Abnormitäten. So können auf MRTs-Scans gebrochene Knochen, innere Verletzungen oder Schlaganfälle erkannt und beurteilt werden. Im Fall von Schlaganfällen, ist hierbei nicht nur eine Diagnose möglich sondern sogar eine Prävention. 
% Wie genau?
Um auf dem MRT-Scan des Gehirns eines Patienten einen Bereich zu entdecken, der auf einen Schlaganfall hindeuten könnte studiert ein Neurologe die einzelnen Bilder genau. Hierbei ist für den Behandlungserfolg neben der Qualität der Bilder vor allem die Methode der Anschauung wichtig. Diese sollte für den Arzt so intutiv und wie möglich und auf seine Bedürfnisse zugeschnitten sein. 
Diese Arbeit untersucht den Einsatz von AR-Headsets in der Verarbeitung von MRT-Scans zur Schlagabfallprävention.

- Neurologen sehen sich MRT Scans Bild für Bild an
- Schlaganfälle sind auf Scans sichtbar

\section{Motivation}

Um MRT-Scans zu studieren benutzen Ärzte in der Regel speziell dafür entwickelte Software. Diese stellt das Gehirn meist aus der Sicht von drei verschiedenen Achsen dar, sodass es von allen Seiten zusehen ist. Auf diesen Achsen kann der Nutzer die Ansicht dann durch die Schichten des Scans bewegen. Die aktuelle Position der gerade angezeigten Schicht wird in jeder der anderen Achsenansichten farbig eingezeichnet, um den Nutzer ein möglichst umfassendes Bild des gescannten Gehirns zu vermitteln. 
Ein Beispiel für die Oberfläche solch einer eben Beschriebenen Software ist in Abbildung \ref{images/mrtSoftware} zu sehen. 
% Grundlagen?

Obwohl Anwendungen dieser Art wohl am meist verbreitetsten sind, um MRT-Scans zu analysieren 
%Refrerenz?
weisen sie einige Unzulänglichkeiten auf.
Zum Einen entsteht durch die Reduzierung um eine Dimension ein verzerrtes Bild des Gehirns. Die Darstellung der verschiedenen Achsen auf den Scans soll den Arzt bei der Orientierung unterstützen. Trotzdem muss dieser immer seinen Fokus zwischen zwei oder mehr Bildern wechseln und über ein gewisses räumliches Vorstellungsvermögen besitzen, um das Gesamtbild des Gehirns in seinem Geist zu rekonstruieren. Dies ist ein kognitiver Aufwand, der Ärzte zusätzlich belastet, während sie sich darauf konzentrieren Anomalien in den Scans eines Patienten zu erkennen und einzuschätzen. 
Wird auch einem Scan ein Schlaganfall entdeckt, ist es durch die Abstraktion des Organs außerdem schwierig eine korrekte Vorstellung von der Größe und Lage des betroffenen Bereichs zu bekommen, da jeweils nur eine Schicht des Gehirns sichtbar ist. 
Hier würde eine dreidimensionale Ansicht des gescannten Gehirns einen sehr viel deutlicheren Einblick in den Zustand des Patienten liefern. 

Es bietet sich die Möglichkeit moderne Technologie einzusetzen, um die Arbeit von Neurologen zu erleichtern und ihre Ergebnisse und somit die Gesundheit ihrer Patienten zu verbessern.


- kein intuitiver umgang, kein "Spaß" Faktor
- 2D Darstellung bietet oft keine Vorstellung davon, wie groß der Schlaganfallbereich ist
- AR leichter zugänglich als komplizierte Software

Anwendung:
- Verdeutlicht Größe und Lage
- Macht betrachten der Daten interessanter
- Kann Erfolg von Therapie visualisieren
-> kann zu Lehrzwecken dienen
- Kann helfen dem Patienten zu veranschaulichen
- 

- UX -> intuitiver, besseres Verständnis

- Ar kommt schon in Medizin zum Einsatz (Referenz)
- Ärzte fänden AR unterstütztes Arbeiten hilfreich (Referenz)

Die Anwendung ist nicht als einsetzbares Produkt zu verstehen sondern eher als Prototyp, der die Nützlichkeit und das Potenzial des Programms beweisen soll.

\section{Zielsetzung}

\section{Struktur dieser Arbeit}

