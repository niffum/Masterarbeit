% Anforderungsanalyse

\chapter{Anforderungsanalyse}
\label{anforderung}

\section{Interviews}

Zur Bestimmung der Anforderungen, die von der Anwendung erfüllt werden sollen wurden iterativ mehrere Interviews mit dem betreuenden Neurologen der Charité geführt.

Wie bereits beschrieben, ist es das Ziel dieser Arbeit Neurologen eine dreidimensionale Darstellung von Gehirn-MRT-Scans in einer interaktiven AR- bzw. VR-Anwendung zur Verfügung zu stellen, um zu untersuchen, inwiefern diese die Ärzte in ihrer Arbeit besser unterstützt, als bisher verwendete Darstellungssoftware. Die hypothetischen Vorteile eines solchen Programmes wurden im Kapitel \ref{motivation} erläutert.
Für den Vergleich mit bisheriger Software ist eine Anwendung prototypischen Charakters ausreichend. Diese soll auf die grundlegenden Interaktionen beschränkt sein, die ein Neurologe für die Verarbeitung der Daten benötigt.

Der Fokus der Untersuchung liegt auf dem Bereich der akuten Schlaganfall Behandlung. Wie im Kapitel \ref{grundlagen} beschrieben untersucht der Neurologe hierfür die Abbildungen des Gehirns eines Patienten, die durch den MRT-Scan erzeugt wurden. Dabei hält er nach Bereichen Ausschau, die von einem Schlaganfall betroffen sein könnten, um die Art des Schlaganfalls zu bestimmen, ? sowie den entstandenen Schaden einschätzen zu können. Betroffene Bereiche werden anschließend markiert. Die Markierung kann danach ein- und ausgeblendet werden.

% ZU MOTIVATION
Weiterhin könnte man anhand des Modells auch Voraussagen von Therapieerfolgen anschaulich demonstrieren.
%%

Folgende Anforderungen haben sich aus den Interviews des Neurolgen ergeben, die erfüllt sein sollten, um den erfolgreichen Einsatz von mARt zu ermöglichen.

\subsection{Anforderungen aus Interviews}
Die aus dem Interview ermittelten Anforderungen wurden in der folgenden Tabelle aufgelistet. Zur besseren Referenzierung wurden sie jeweils mit Bezeichner versehen. Weiterhin wurde die Priorität jeder Anforderungen auf niedrig, mittel, oder hoch geschätzt.

\begin{table}
\begin{tabular}{|c|p{10cm}|c|}
\hline
Anforderungsbezeichnung & Anforderung & Priorität \\
\hline
A01 & 3D Darstellung der MRT Daten & hoch\\
\hline
A02 & Einblenden des gekennzeichneten Bereichs in 3D & hoch\\
\hline
A03 & Scrollen durch Schichten der 3D-Darstellung auf mindestens einer Achse& mittel\\
\hline
A04 & Erkennbarkeit der inneren Struktur des Gehirns (entsprechend der MRT-Bilder) & hoch\\
\hline
A05 & Scrollbare 2D Darstellung der MRT-Bilder auf mindestens einer Achse & mittel\\
\hline
A06 & Kontrast und Helligkeit sollten einstellbar sein & hoch\\
\hline
A07 & Man sollte die Darstellung vergrößern können & hoch\\
\hline
A08 & Die Darstellung sollte verschiebbar sein & hoch\\
\hline
A09 & Interaktionselemente sollten die Darstellung nicht verdecken & niedrig\\
\hline
A10 & Scrollen durch Verwendung eines Scrollrads & niedrig\\
\hline
A11 & Auswahl verschiedener MRT-Sequenzen (falls vorhanden) & niedrig\\
\hline
A12 & Unterstützung von nifti-Daten & mittel\\
\hline
A13 & Verschiedene Ansichten können nebeneinander dargestellt werden. (z.B. vor und nach Therapie) & mittel\\
\hline

\end{tabular}
\caption{\label{tab:table-name} Durch Interview bestimmte Anforderungen.}
\end{table}

\section{User Stories}

Anhand dieser Anforderungen wurden User Stories formuliert. 
%Referenz User Stories

\begin{table}
\begin{tabular}{|c|p{10cm}|c|}
\hline
Story-Bezeichnung & User Story & Anforderungsreferenz \\
\hline
U01 & Als Nutzer möchte ich ein 3D-Modell des gescannten Gehirns sehen, um ... & A01\\
\hline
U02 & Als Nutzer möchte ich den gekennzeichneten Bereich innerhalb des Gehirn sehen können, um einzuschätzen wie groß der Bereich tatsächlich ist.  & A02\\
\hline
U03  & Als Nutzer möchte ich den gekennzeichneten Bereich ein- und ausblenden können, um mich auf diesen, oder die Darstellung an sich konzentrieren zu können.  & A02\\
\hline
U04 & Als Nutzer möchte ich auch dann noch die Strukturen des Gehirns erkennen, wenn der gekennzeichnete Bereich eingeblendet ist. & A03\\
\hline
U05 & Als Nutzer möchte ich eine möglichst genaue und gut erkennbare Abbildung der MRT-Bilder, damit ich eine Vorstellung habe, wie das Gehirn des Patienten aussieht. & A05\\
\hline
U06 & Als Nutzer möchte ich durch das 3D-Modelll scrollen können, um es vergleichbar mit der 2D Darstellung zu verwenden. & A04\\
\hline
U07 & Als Nutzer möchte ich die MRT-Darstellungen frei im Raum bewegen können, um sie meiner Position im Raum anzupassen. & A07\\
\hline
U08 & Als Nutzer möchte ich die MRT-Darstellungen skalieren können, um die Darstellung gut zu erkennen. & A08\\
\hline
U09 & Als Nutzer möchte ich die MRT-Darstellungen drehen können, um den besten Blickwinkel auf den für mich relevanten Bereich zu bekommen. & A09\\
\hline
U10 & Als Nutzer will ich, dass die Darstellung nicht von anderen Elementen verdeckt wird, damit ich sie uneingeschränkt sehen kann. & A10\\
\hline
U11 & Als Nutzer möchte ich mit einem Scrollrad durch die Darstellung scrollen können, um genaue Kontrolle darüber zu haben, welche Schichten angezeigt werden. & A11\\
\hline
U12 & Als Nutzer möchte ich alle vorhandenen MRT-Sequenzen sehen und zwischen ihnen wählen können, damit ich alle notwendigen Informationen zu dem Scan nutzen kann. & A12\\
\hline
U13 & Als Nutzer möchte ich Dateien im nifti-Format in der Anwendung verwenden können, damit ich sie nicht vorher umwandeln muss. & A13\\
\hline
U14 & Als Nutzer möchte ich aus aus verschiedenen Darstellungen wählen können, die nebeneinander angezeigt werden, um direkte Vergleiche zwischen diesen ziehen zu können. & A14\\
\hline

\end{tabular}
\caption{\label{tab:table-name}Aus Anforderungen abgeleitete User Stories.}
\end{table}

\section{Technische Anforderungen}
% Anforderungen, die ich aus User Stories ableite
% UX ist hier wichtig