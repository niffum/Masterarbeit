% Anforderungsanalyse

\chapter{Anforderungsanalyse}
\label{anforderung}

\section{Interviews}

Wie bereits beschrieben, ist es das Ziel dieser Arbeit Neurologen eine dreidimensionale Darstellung von Gehirn-MRT-Scans in einer interaktiven AR- bzw. VR-Anwendung zur Verfügung zu stellen, um zu untersuchen, inwiefern diese die Ärzte in ihrer Arbeit besser unterstützt, als bisher verwendete Darstellungssoftware. Die hypothetischen Vorteile eines solchen Programmes wurden im Kapitel \ref{motivation} erläutert.
Für den Vergleich mit bisheriger Software ist eine Anwendung prototypischen Charakters ausreichend. Diese soll auf die grundlegenden Interaktionen beschränkt sein, die ein Neurologe für die Verarbeitung der Daten benötigt.

Der Fokus der Untersuchung liegt auf dem Bereich der akuten Schlaganfallbehandlung. Wie im Kapitel \ref{grundlagen} beschrieben untersucht der Neurologe hierfür die Abbildungen des Gehirns eines Patienten, die durch den MRT-Scan erzeugt wurden. Dabei hält er nach Bereichen Ausschau, die von einem Schlaganfall betroffen sein könnten, um die Art des Schlaganfalls zu bestimmen, ? sowie den entstandenen Schaden einschätzen zu können. Betroffene Bereiche werden anschließend vom Arzt markiert. Diese Markierung kann danach ein- und ausgeblendet werden.

Um zu bestimmen, welche Interaktionen und Funktionen ein Neurologe benötigt, um effektiv mit der Anwendung arbeiten zu können, wurden iterativ mehrere Interviews mit dem betreuenden Neurologen der Charité Berlin geführt. Die Antworten des Arztes wurden in Anforderungen übertragen, die im folgenden Abschnitt festgehalten wurden. 

\section{Anforderungen als User Stories}

Die im Gespräch mit dem Arzt entwickelten Anforderungen an mARt wurden in Tabelle \ref{tab:userStories} als User Stories.
User Stories werden in der Software-Entwicklung verwendet, um einzelne Funktionen zu definieren, die möglichst unabhängig voneinander implementiert werden können sowie mehr aus Sicht des Nutzers zu denken, statt aus der Sicht des Entwicklers \cite{UserStoriesApplied}.
Zur besseren Referenzierung wurden die Stories jeweils mit Bezeichnern versehen. Da viel der Interaktionen sowohl für die zwei- als auch dreidimensionale Darstellung benötigt werden wurde weiterhin gekennzeichnet, welche Darstellung von der jeweiligen Story betroffen ist.

\begin{longtable} {p{.125\textwidth}p{.50\textwidth}p{.05\textwidth}p{.225\textwidth}}
\toprule
Bezeichnung & User Story & 2D/3D & Anforderungsreferenz \\
\toprule
U01 & Als Nutzer möchte ich eine möglichst genaue und gut erkennbare Abbildung der MRT-Bilder, damit ich eine Vorstellung habe, wie das Gehirn des Patienten aussieht. & beide & A05\\
\midrule 
U02 & Als Nutzer möchte ich die Daten als zweidimensionale Schichten aus mindestens einem Winkel betrachten können, um diese wie gewohnt untersuchen zu können.  & 2D & A02\\
\midrule 
U03 & Als Nutzer möchte ich die Daten als dreidimensionales Volumen betrachten können, um ein besseres Verständnis für den Zustand des Gehirns zu erhalten. & 3D & A02\\
\midrule 
U04 & Als Nutzer möchte ich mindestens zwei Datensätze nebeneinander anzeigen lassen, um direkte Vergleiche zwischen diesen ziehen zu können. & beide & A14\\
\midrule 
U05 & Als Nutzer möchte ich aus verschiedenen Datensätzen auswählen können, welche angezeigt werden, um mich auf die relevanten Daten konzentrieren zu können. & beide & A01\\
\midrule
U06 & Als Nutzer möchte ich entscheiden können, ob ich mehrere angezeigte Datensätze gleichzeitig oder einzeln manipuliere (z.B. Kontrast), um die Einstellungen genau an jeden Datensatz anpassen zu können. & beide & A14\\
\midrule 
U07 & Als Nutzer möchte ich auf mindestens einer Achse durch beide Darstellungen scrollen können, um das Innere des Organs zu untersuchen. & beide & A04\\
\midrule 
U08 & Als Nutzer möchte ich mit einem Scrollrad durch die Darstellungen scrollen können, um genaue Kontrolle darüber zu haben, welche Schichten angezeigt werden. & beide & A11\\
\midrule 
U09 & Als Nutzer möchte ich zwischen einer zwei- und dreidimensionalen Darstellung der Daten wechseln können, um die Vorteile beider Darstellungen für meine Untersuchung zu nutzen. & beide & A02\\
\midrule 
U10 & Als Nutzer möchte ich, dass bisherige Manipulationen beim Wechsel zwischen 2D und 3D übernommen werden, um die Orientierung und Sichtbarkeit zu erhalten. & beide & A13\\
\midrule 
U11  & Als Nutzer möchte ich den gekennzeichneten vom Schlaganfall betroffenen Bereich ein- und ausblenden können, um mich auf diesen, oder die Darstellung an sich konzentrieren zu können. & beide & A02\\
\midrule
U12 & Als Nutzer möchte ich auch dann noch die Strukturen des Gehirns erkennen, wenn der gekennzeichnete Bereich eingeblendet ist. & beide & A03\\
\midrule 
U13 & Als Nutzer möchte ich die MRT-Darstellungen drehen können, um den besten Blickwinkel auf den für mich relevanten Bereich zu bekommen. & 3D & A09\\
\midrule 
U14 & Als Nutzer möchte ich die Helligkeit und den Kontrast der Darstellung verändern können, um Strukturen besser deutlich zu machen. & beide & A09\\
\midrule 
U15 & Als Nutzer möchte ich die MRT-Darstellungen skalieren können, um die Darstellung gut zu erkennen. & beide & A08\\
\midrule 
U16 & Als Nutzer möchte ich die MRT-Darstellungen frei im Raum bewegen können, um sie meiner Position anzupassen. & beide & A07\\
\midrule 
U17 & Als Nutzer möchte ich, dass das Hologram seine Position im Raum behält, um es von allen Seiten betrachten zu können. & beide & A13\\
\midrule 
U18 & Als Nutzer möchte ich alle vorhandenen MRT-Sequenzen sehen und zwischen ihnen wählen können, damit ich alle notwendigen Informationen zu dem Scan nutzen kann. & beide & A12\\
\midrule 
U19 & Als Nutzer möchte ich Dateien im NIfTI-Format in der Anwendung verwenden können, damit ich sie nicht vorher umwandeln muss. & beide & A13\\
\midrule 
U20 & Als Nutzer möchte ich Dateien im DICOM-Format in der Anwendung verwenden können, damit ich sie nicht vorher umwandeln muss. & beide & A13\\

\bottomrule
\caption{\label{tab:userStories}Aus Interviews abgeleitete User Stories.}
\end{longtable}

